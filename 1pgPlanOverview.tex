\documentclass[10pt]{article}

% Packages
\usepackage[margin=1in]{geometry}   % 1 inch margins
\usepackage{fancyhdr}               % Custom headers/footers
\usepackage{lastpage}               % Reference total page number
\usepackage{graphicx}
\usepackage{caption}
\usepackage{comment}
\usepackage{subcaption}
\usepackage{titlesec}
\usepackage{array}
%\usepackage{hyperref}
\usepackage[most]{tcolorbox}
\usepackage{cleveref}
\usepackage{pgfgantt}
\usepackage{adjustbox} % adjust the side of the gant chart
\usepackage{multicol}
%%%%
\usepackage{booktabs} % used for \toprule in tables

\usepackage{subcaption}
\usepackage{multicol}
\usepackage{enumitem}
\usepackage{tikz}
\usetikzlibrary{positioning}

%\usetikzlibrary{decorations.pathmorphing} % example library
\usetikzlibrary{shapes, arrows.meta, positioning, decorations.pathreplacing, calc, fit}



\usepackage[numbers,sort]{natbib} % Cite things in order in the body
%% Removed from the data managemnt page
\usepackage{times}
\usepackage{url}
\usepackage{color} % needed for todo
\usepackage{enumitem}
\usepackage{soul} % highlighting 
\urlstyle{same} % Have the URL font be identical to the rest of the paper


%\setlist{leftmargin=6.0mm, noitemsep}
\setlist{noitemsep}

\usepackage{xspace} % Needed for et al.
\newcommand{\ie}{\emph{i.e.,}\xspace}
\newcommand{\eg}{\emph{e.g.,}\xspace}
\newcommand{\etc}{etc.\xspace}
\newcommand{\etal}{\emph{et~al.}\xspace} 
\newcommand{\todo}[1]{\textcolor{cyan}{\textbf{[#1]}}}
\newcommand{\dan}[1]{\textcolor{blue}{{\it [Dan: #1]}}}
\newcommand{\sam}[1]{\textcolor{red}{{\it [Sam: #1]}}}



\newcommand\MainTitle{CISE Paper Submission: MABS \& QDNs}
%\newcommand\MainTitle{XXXXX}

\setlist{noitemsep, leftmargin=6.0mm}
\newcommand{\smallTitle}[1]{\vspace{1mm} \noindent \textbf{#1: }}
\newcommand{\descStep}[2]{\noindent \textbf{#1: } #2}



% Set normal paragraph indentation (e.g., 15pt)
\setlength{\parindent}{15pt}

% Ensure the first paragraph after section titles is NOT indented
% but subsequent paragraphs ARE indented

\makeatletter


% RE-ENABLE THIS
\let\orig@section\section
\renewcommand{\section}{%
  \@ifstar{\orig@section*}{\@section@noindent}%
}
\newcommand{\@section@noindent}[1]{%
  \orig@section{#1}%
  \@afterindentfalse\@afterheading
}
\let\orig@subsection\subsection
\renewcommand{\subsection}{%
  \@ifstar{\orig@subsection*}{\@subsection@noindent}%
}
\newcommand{\@subsection@noindent}[1]{%
  \orig@subsection{#1}%
  \@afterindentfalse\@afterheading
}
\let\orig@subsubsection\subsubsection
\renewcommand{\subsubsection}{%
  \@ifstar{\orig@subsubsection*}{\@subsubsection@noindent}%
}
\newcommand{\@subsubsection@noindent}[1]{%
  \orig@subsubsection{#1}%
  \@afterindentfalse\@afterheading
}

\titlespacing\section{0pt}{3pt plus 1pt minus 1pt}{0.75pt plus 0.5pt minus 0.5pt}
\titlespacing\subsection{0pt}{3pt plus 1pt minus 1pt}{0pt plus 0.5pt minus 0.5pt}
\titlespacing\subsubsection{0pt}{3pt plus 1pt minus 1pt}{0pt plus 0.5pt minus 0.5pt}

\usepackage{tikz}
\usetikzlibrary{shapes.geometric, arrows.meta, positioning}

% Header and footer setup
\pagestyle{fancy}
\fancyhf{} % Clear all header/footer fields

% Define values for the header

\fancyhead[L]{LLLLLL}
\fancyhead[C]{}
\fancyhead[R]{dxkvse@rit.edu}

% Footer: Page x of y
%\fancyfoot[C]{Page \thepage\ of \pageref{lastpage}} % Comment out to remove page number

% Title
\title{\vspace{-2cm} \bfseries \MainTitle}%: Integrating DRL, CMAB, and Post-Processing Strategies}
%\author{Daniel Krutz \{dxkvse@rit.edu\}\\}
\date{}

%% Change the size of the section labels
\titleformat{\section}
  {\normalfont\bfseries\Large} % font and size %large
  {\thesection}{1em}{}              % section number formatting

%% Change the spacing between sections
\titlespacing*{\section}
  {0pt}   % Left margin
  {1ex}   % Space before the section
  {0.5ex} % Space after the section

\begin{document}

%% start defining the layout of the 1st page
\fancypagestyle{firstpage}{
  \fancyhf{}              % Clear all header/footer
  \renewcommand{\headrulewidth}{0pt} % No header rule
%  \fancyfoot[C]{Page \thepage\ of \pageref{lastpage}}   % Comment out to remove page number
}
%% end defining the layout of the 1st page

\maketitle
%\thispagestyle{fancy} % Apply fancy header/footer to title page
%\thispagestyle{empty} % Suppress header/footer on the first page
\thispagestyle{firstpage}

\linespread{1.005} % Help with NSF page spacing error
\selectfont

\vspace{-8mm}
{\par\centering
 \begin{tcolorbox}[enhanced, width=1.03\linewidth, 
            %colback=blue!50!white!20,
            arc=0pt, outer arc=0pt, 
            borderline={1.5pt}{0pt}{black!90},
            borderline={0.25pt}{3pt}{black!70, sharp corners},
            drop fuzzy shadow]
    \centering
{\large %\textbf{Synopsis}
\par\medskip}
\normalsize%\itshape
Use novel MAB-focused ML to Assist Stochastic Entanglement Path Selection and ?QubitAllocation?
\end{tcolorbox}\par}


%%% Are we going to address Jie's comments in the email
% I’m not sure that simply adding a new ML method is the best approach. Bandits are already ML-based, so unless this new approach tackles a specific quantum network challenge connected to path selection, it might pull the proposal too far toward ML. It would be great if the ML aspect were designed to solve a unique quantum problem, but I don’t have a good idea for that yet.


% Adaptive/Online Entanglement Routing

\section*{Overview}

The objective of this work is to apply MAB-based techniques to enhance entanglement routing in QDNs. This will be achieved by xyz...



\section{Open Problems in Adaptive/Online Entanglement Routing}


% DK: How to have the ML address these open problems
\begin{enumerate}
    \item Probabilistic Link Success and Uncertainty - Quantifying uncertainty in real-time and propagating it through routing decisions; Handling correlated failures (e.g., multiple links affected by the same noise source); Incorporating imperfect fidelity and probabilistic purification in path evaluation.
    \item 
    \item 
    \item 
    \item 

\end{enumerate}

? Entangled pairs may require purification to improve fidelity, consuming additional qubits and time. < Do something with our Error Mitigation work? - How could error mitigation be used to address these challenges?


> What is the unique quantum problem that we are solving?

\smallTitle{Problems Addressed}

\begin{enumerate}
    \item How to support entanglement routing (\eg selecting a path, allocating qubits, scheduling entanglement attempts, performing entanglement swapping, handling purification, managing memory constraints) in stochastic QDNs.
    \item 
    \item -- Something ML focused.
\end{enumerate}




\smallTitle{Research Questions}
\begin{enumerate}
    \item 
    
%    \item \descStep{XXX}{XXX}
%    \item \descStep{XXX}{XXX}
%    \item \descStep{XXX}{XXX}
%    \item \descStep{XXX}{XXX}
%    \item \descStep{XXX}{XXX}

\end{enumerate}

% ? Something about topology? - Does it make sense to do this?
% Entanglement routing techniques
% ML-focused RQs



\smallTitle{Advancements Over State of the Art}
\begin{enumerate}
    \item \descStep{XXX}{XXX}
    \item \descStep{XXX}{XXX}
    \item \descStep{XXX}{XXX}
    \item \descStep{XXX}{XXX}
    \item \descStep{XXX}{XXX}

    
\end{enumerate}



\smallTitle{How the ML is designed to solve a unique quantum problem}







\label{lastpage}
\pagebreak
%\cleardoublepage


%\appendix


%% DK: Put onto a different page since it does not count against the page limit
%\setcounter{page}{1}

\cfoot{\thepage}
\pagenumbering{roman}


\end{document}




What our work will do
What makes the work unique/different